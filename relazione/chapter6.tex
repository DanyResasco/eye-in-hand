%!TEX root = pag0.tex
\chapter{Conclusioni}
\label{chapter6}
Come visto nel capitolo ~\ref{chapter5} entrambi i metodi comportano un minimo errore di stima della posizione. Questo errore di può essere causato da:
\begin{itemize}
\item nella fase di inizializzazione di Ptam. Non si raggiungono abbastanza features per creare una griglia stabile
\item nello spostamento per la stima della scala. Non si raggiungono o si superano i cm richiesti
\item nella fase di riproiezione dei punti.
\item ambiente circostante. 
\end{itemize}
Per rendere più stabile l'algoritmo si potrebbe utilizzare la dinamica del manipolare e un ulteriore controllo durante il percorso per correggere l'eventuale errore di posa.
